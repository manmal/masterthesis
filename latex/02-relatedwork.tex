\section{Related Work}

In this chapter, the reader will be introduced to the proceedings in science which are
relevant or related to this thesis. They are grouped into the following topics of interest:

\begin{itemize}
	\item \textbf {Motivation for this thesis} - Provides an explanation on why the chosen 
		  topic of this thesis is generally of interest, and why similarity measures in music are feasible
		  (given optimal circumstances).
	\item \textbf {Features of digital music} - Gives an overview of existing methods of 
		  feature extraction which are purely based on computational methods.
	\item \textbf {Subjective artist similarity computation} - Lists literature which is related
		  to this thesis' problem of computing the subjective similarity of artists, which is
		  based on subjectiveness as experienced by humans.
	\item \textbf {Visualization of artist similarity} - Provides an overview of existing 
		  methods and models of visualization of data similarity in general, and music similarity
		  in particular.
	\item \textbf {Drawing and optimization of graphs based on multidimensional scaling} - Gives an
		  overview of literature dealing with the technical and mathematical background of 
		  multi-dimensional visualization on two-dimensional output devices.
\end{itemize}

\subsection{Motivation for the Topic of This Thesis}

Music is an integral part of the daily life in nearly all societies, and the list of published titles 
is growing every day. As huge amounts of data tend to be hard to digest, ontologies have to be created, 
by which music can be categorized in a hierarchical fashion. Aside from the author's motivation of 
choosing the topic of this thesis, a great interest in music classification can be observed in scientific 
literature. This is related to the problem that the categorization of arbitrary music titles
is neither implicit nor trivial.
Serving the demands of Electronic Music Distribution (EMD), the authors of \cite{pachet:02g} elaborate
on the feasability of music similarity measures. It is found in \cite{pachet:02g} that the introduced
similarity measure (timbre similarity) combined with other measures can yield interesting results. It is
also mentioned that the interpretation of experimental results in the field of music similarity is challenging
due to the subjective demands. 

It is clear that even the best-educated music experts could hardly agree on 
any distinct similarity measure between two music titles, due to the implicit fuzziness of subjective measures.
It can be assumed that it is rare that two humans would agree on the same similarity between music files
if they vote independent of each other.

\subsection{Features of Digital Music}

As opposed to subjective artist similarity, there are music features or measures which can be retrieved by
purely computational approaches. In the field of audio feature extraction, a wide range of classifiers
(feature extractors) has been created. These classifiers in many cases run a bitstream analysis of a digitally
stored music file and extract one or more reproducible measures characterizing the file. 



TODO: MORE SOURCES FOR GENERAL FEATURE EXTRACTION



Interestingly, it is confirmed in \cite{LID_05ismir} that the use of psycho-acoustic enhancements before
feature extraction improves the classification accuracy significantly. It can be concluded that the outcomes of
audio feature extraction are influenced by many factors which are not always intuitive.
As has been mentioned previously, most audio classifiers analyze the bitstream of music files - however,
the bitstream is only one dimension of a piece of music, if we regard it as a multidimensional object. For example,
it is also possible to analyse the lyrics, as has been done in \cite{DBLP:conf/ismir/MayerNR08}.

\subsection{Subjective Artist Similarity Computation}

Subjective similarity, as the author understands it, expresses human opinions on a certain object. As previously
mentioned, it is obvious that humans will hardly agree on attributes of music, and the same person might even make
different statements in the course of time, depending e.g. on her mood. The following applies to both artist 
similarities and music file similarities, since the former may be constructed from aggregations of the latter 
(it has to be noted at this point that many artists tend to produce music from multiple genres, thus making
an artist-to-artist-comparison difficult or even infeasible).
In article \cite{Ellis02thequest} it is found that it is doubtable that a common ground truth for subjective
artist similarity even exists, because of the inhomegeneity of measures made by the involved users. It can be
deduced that a meaningful model of subjective music similarity will in most cases only resemble a compromise
between different stakeholders.
As inferred from \cite{Berenzweig03alarge-scale} and \cite{mcfee09_hesas} there are different approaches to 
retrieving a model of subjective similarity for a given set of music files, which include:

\begin{itemize}
	\item Conduction of surveys with end users
	\item Opinions of experts
	\item Co-occurrence of files in end users' libraries or playlists
	\item Data mining of text in web sources, as performed in \cite{Whitman02inferringdescriptions}
	\item Leveraging data gathered by social music services
\end{itemize}

As it is intended by this thesis to provide a concept for a fast and fully automatic approach to similarity
measuring, we will concentrate on the last approach, the usage of data provided by social music services.
Hybrid computation methods, such as the method described by \cite{mcfee09_hesas} (combining acoustic 
features with text excerpts and tags retrieved from online services) turn out to be hardly feasible on a 
mobile device because of performance requirements. It is assumed by the author that for a rough estimation 
of music file or artist similarity, the data provided by social music services (as opposed to hybrid 
approaches) is sufficiently meaningful, as their daily user base is in the millions and still growing.

A combination or fusion of similarity rankings from various social music services has been performed by the
author of \cite{Marshall:2010}. In this article it is demonstrated that various methods of embedding or fusing
similarity rankings from online services can provide different meaningful similarity models, some of which
give more weight to unknown artists. However, this approach is clearly limited to the embedding of rankings 
and does not compute continuous values as similarity measure.

\subsection{Visualization of Artist Similarity}

\subsection{Drawing and Optimization of Graphs Based on Multidimensional Scaling}


\subsection{Summary of this Section}