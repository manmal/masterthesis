In this chapter, the scope of this thesis will be defined and a user scenario outlined. 

\subsection{Scope Definition}

The scope of this thesis is defined by the following goals: 

\begin{itemize}
	\item Verification of the selected artist similarity computation method being feasible for mobile end user devices.
	\item Verification of the selected artist similarity visualization method being a sensible choice for mobile end user devices.
	\item Description of the implementation of a prototype (able to perform the selected computation and visualization methods) on the Android platform.
	\item Description of the design and presentation of the results of a user study.
\end{itemize}


\subsection{Selected Artist Similarity Computation}

After consideration of the options for similarity computation in section ~\ref{ch:relatedwork}, the author has concluded that the approach presented in \cite{Morrison:2003:FMS} (combining multi-dimensional scaling with spring models and interpolation), seems to be a promising approach to the problem of music library visualization, accommodating mobile devices by especially fast computation. The approach will not be applied unaltered, but modifications and enhancements will be made and described in this thesis. 
It must be noted that this computational method is limited in the amount of objects which can feasibly be displayed (and computed) on a mobile device. Also, since for some data structures no similarity metrics are currently available, the adapted MDS method is not suitable for all kinds of music data. Therefore, a fallback algorithm is selected for the display of hierarchical data objects: the force-directed layout algorithm presented in \cite{Kobourov04} is of low computational complexity, and its results seem promising.

\subsection{Selected Visualization Computation}

Since an MDS computation approach has been selected for further proceeding, the visualization computation is entangled with the similarity computation and cannot be freely chosen. Also, the presentation space for visualization is chosen to be two-dimensional, since three-dimensional visualizations are hard to implement and are unlikely to be displayed as intended on mobile devices (with many objects being presented). Thus, the visualization method for MDS-generated object clusters is constrained to be a two-dimensional layout algorithm, positioning the laid-out objects such that they resemble their MDS-generated coordinates. For ease of navigation, zooming and panning will be added, and different colors will be used for faster identification of object types.

\subsection{Selected Algorithm for Removal of Overlapping of Artists}

The author of this thesis decided to use an iterative approach to the removal of graph node overlappings - a modification and simplification of the idea of the force-transfer algorithm \cite{Huang03force-transfer:a}. Graph nodes are added into a fresh space, one by one, and while they are added, they are positioned such that overlappings with other nodes are eliminated. This elimination is performed by moving the freshly added node on the vector [overlapped node's center TO added node's center] so far that they don't overlap anymore. As this might create new overlappings, several iterations of this "'pushing away"' are performed, until the freshly added node does not overlap any other nodes. The outcome of this algorithm is a cleanly laid out representation of the previously crowded MDS computation outcome.

\subsection{Selected Layout Algorithm for Artist Discovery}

Since the graph of discovered artists is not star-shaped (one single subject artist circled by discovered artists), but has multiple main nodes (potentially multiple subject artists circled by their discovered artists), it would be too complex to calculate artist positions deterministically. Instead, a force-directed layout algorithm is chosen, where all graph nodes push away from each other, but the secondary nodes (discovered artists) are attracted by their primary node (subject artist). These forces are applied continuously in iterations, as long as a certain movement threshold has not been undercut. With only one subject artist, such a graph will be star-shaped, with the discovered artists circling the subject. After the optimal parameters for these forces have been found, such an algorithm can handle an arbitrary number of displayed artists.

\subsection{Overview of the Assembly of Algorithms and Information Flow}

To give the reader a better picture of the sections to come, an overview shall be given which explains how the selected algorithms process and pass on information to each other. An in-depth explanation of these information flows will be given in the section ~\ref{ch:implementation}.

\subsubsection{Library Visualization}

DURCH GRAFIK ERSETZEN:

\begin{itemize}
	\item Extraction of music metadata on the device
	\item Matching of the device's music metadata with metadata from web sources
	\item Querying of Artist Similarity data from web sources
	\item Completion of Artist Similarity data
	\item Laying out artists in 2D space with a Multi Dimensional Scaling (MDS) algorithm
		\subitem Building up of a distance matrix between artists
		\subitem Generation of a subset of artists and laying them out according to spring model forces
		\subitem Addition of the remaining artists, positioning them around the initial subset
		\subitem Application of spring model forces on all nodes for a few iterations
	\item Removal of overlapping of artists' depictions in 2D space
	\item Display of the laid out artists in OpenGL
	\item Continuous reaction to user actions (zooming, panning, tapping) 
\end{itemize}

\subsubsection{Artist Discovery}

Artist discovery is initiated by the user selecting a certain artist ("'subject"'), and requesting discovery mode.

DURCH GRAFIK ERSETZEN:

\begin{itemize}
	\item Querying of the artists most similar to "'subject"' from web sources
	\item Integration of the retrieved artists around "'subject"' in 2D space, at randomized but similar positions
	\item Continuous re-arrangement of the retrieved artists based on a force-based layout algorithm (also reacting to newly added similar artists)
\end{itemize}

\subsection{Summary of this Section}