\chapter*{Abstract}

Music and its interpretation has long been an important part of human culture. Technical advancements and resulting technologies have made playback of previously recorded sounds possible, and playback has also become an integral cultural component of high importance. Digitalization of music recordings has not only introduced superior sound quality to everybody's living room, but has also enabled transmission of large music collections over the internet in the matter of minutes.
This advancements resulted in a new class of business models which provide customers with huge music collections via streaming technology, for a considerably small monthly fee. Since user interfaces for music playback software have not changed much in the last years, they are not necessarily optimized for search and exploration in collections of millions of music files. The scope of this master's thesis is to investigate a promising ordered visualization of music in 2D space - through utilization of artist similarity - for mobile applications.
Whether the chosen method to compute music similarity (multi-dimensional scaling) can be used feasibly in the context of mobile apps, and whether 2D visualization of music collections is well suited for search and exploration of objects therein, are the main questions this thesis will concentrate on.

These questions are investigated on their techical feasibility by finding and optimizing methods for the efficient retrieval of music similarity, and through the construction of an Android app prototype. Subsequently, a user study is planned and carried out, comparing important aspects of the prototype with other existing music apps for Android. Among the investigated properties is the question whether the prototype allows for an easier or better forming of a mental model of a music collection. Also, the study tries to find out whether contents can be found quicker within a collection.
Although the number of study participants is too small for statistically significant conclusions, hints and possible explanations can be derived: The implementation of a 2D visualization of music artists (incorporating their similarity) seems to positively affect the users' memorization ability of contained objects. It's possible that a better or more stable mental model can be built in this context. Search and exploration of objects in a music collection seem to not be positively affected by the proposed visualization, as compared to other forms. However, leaving aside list-based visualization forms, the prototype performed comparably well for search and exploration.

It is concluded that the proposed concepts are promising, but that further research is necessary to resolve the problems which have been found.