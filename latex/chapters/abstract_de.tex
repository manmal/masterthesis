\chapter*{Kurzfassung}

Musik und die Intonation derselben ist seit langem ein wichtiger Bestandteil der menschlichen Kultur. Mit dem technischen Fortschritt und den dadurch entstehenden Geräten wurde die Wiedergabe von zuvor aufgenommenen Klängen möglich und wiederum zu einer kulturellen Komponente von zentraler Bedeutung. Die Digitalisierung von Musikaufnahmen sorgte dann nicht nur für exzellente Tonqualität, sondern ermöglichte auch die Kopie von ganzen Musiksammlungen über das Internet in Minutenschnelle. Dadurch bedingt sind in den letzten Jahren neue Geschäftsmodelle entstanden, die dem Benutzer große Mengen an Musiktiteln zur Wiedergabe mittels Streaming-Technologie zur Verfügung stellen, gegen einen vergleichsweise geringen monatlichen Betrag. Da sich die Bedienungsoberflächen für Musikwiedergabe-Software in den letzten Jahren jedoch kaum verändert haben, sind diese nicht für die Suche und Exploration in derart großen Kollektionen optimiert. Die Aufgabenstellung dieser Master's Thesis ist es, eine vielversprechende Anordnung von Musik im 2D-Raum für mobile Applikationen - durch Einbezug der Ähnlichkeit der dargestellten Künstler - zu untersuchen. Die Fragestellungen der Thesis sind, ob die ausgewählte Berechnungsmethode von Musik-Ähnlichkeit (Multi-Dimensional Scaling) in mobilen Apps realistischerweise einsetzbar ist, und ob die 2D-Darstellung der Musiksammlung zum Suchen und Entdecken von Musik gut geeignet ist.

Durch die Findung und Optimierung von Methoden zur effizienten Erlangung von Musik-Ähnlichkeit und Konstruktion einer prototypischen Android App werden diese Fragen auf ihre technische Machbarkeit überprüft. Im Anschluss wird eine Benutzerstudie geplant und durchgeführt, die wichtige Aspekte des Prototyps mit anderen bestehenden Musik-Apps für Android vergleicht. Unter anderem wird erhoben, ob der Prototyp ein leichteres Bilden eines mentalen Modells von der Musiksammlung erlaubt, oder ob Inhalte schneller gefunden werden können. Obwohl die Anzahl der Studienteilnehmer für statistisch signifikante Aussagen zu klein ist, können doch nützliche Schlüsse gezogen werden: 
Die Verwendung einer 2D-Darstellung von Musikkünstlern unter Einbezug ihrer gegenseitigen Ähnlichkeit scheint sich positiv auf das Merkvermögen des Benutzers auszuwirken - es ist möglich, dass dadurch ein aussagekräftigeres mentales Modell gebildet werden kann. Auf das Suchen und Entdecken von scheint diese Darstellungsvariante keine großen Vorteile gegenüber anderen Formen aufzuweisen. Unter Außerachtlassung von reinen Listen-basierten Darstellungsformen scheint die untersuchte Visualisierungsmethode jedoch auch in diesen Aufgabenbereichen vorteilhaft. 

Es wird abschließend festgehalten, dass es sich um ein vielversprechendes Konzept handelt, aber dass weitere Forschung notwendig wäre, um gefundene Probleme zu beheben.