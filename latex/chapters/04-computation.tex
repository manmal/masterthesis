In this chapter, the computation of similarities between artists and construction of layouts based on those similarities will be presented.

\begin{itemize}
	\item \textbf {Matching of music entities from Different Sources} - Describes the methods used for merging music objects (artists, albums, titles) and their similarities, and how they deal with uncertainty regarding similarities.
	\item \textbf {Multidimensional Scaling Algorithm} - Gives details about the MDS algorithm employed for the layout of music collections.
	\item \textbf {Optimizations for Execution on Mobile Devices} - Elaborates on how aforementioned MDS algorithm can be optimized for the execution on devices with relatively low processing power.
	\item \textbf {Downsides of MDS in this Context} - Lists some downsides of the usage of MDS in the given context, as opposed to other layout methods.
\end{itemize}

\section{Matching of Music Entities from Different Sources}

In order to use similarity data from different information sources (here: web services), these data items have to be consolidated in a way that assures that they don't mix up - e.g., objects with the same name are assumed to depict the same artist (typos or naming variations should be amended).

The following approach has been chosen to perform the gathering and matching of different online sources of similarity data:

\begin{enumerate}
	\item A list of music files residing on the mobile device is compiled, and from there a list of locally available music entities (artists, albums, titles) is generated. This list of locally available music entities is not processed, meaning that spelling errors or duplicates are not amended.
	\item All locally available music entities (artists, albums, titles) are linked to corresponding entities in Last.fm's and Echonest's web APIs. This is done by issuing a search query for every locally available music entity to both APIs, and storing the query results on the device (using the query result's list's first item, assuming that it will in most cases resemble the desired music entity). The result is that each locally available music entity is augmented with metadata from both web APIs. Last.fm's representation of each music entity will from there on be used as authoritative metadata source, eliminating spelling errors in the locally available music entity.
	\item For all locally available artists, the Last.fm API is queried, returning a list of at most 100 similar artists for each available artist. This similarity ranking list contains a similarity measure in the range of [0..1] for every similar artist, where 1 means "'most similar or equivalent"' and 0 means "'not related at all"'.
	\item The similarity ranking lists retrieved in the last step are searched to find out whether they contain other locally available artists. This search currently is a case-insensitive equality comparison of artist names. If a locally available artist is found in a similarity ranking list, then the similarity value is stored without further processing, thus creating a similarity connection between one locally available artist and another locally available artist. If it is not foundin the similarity ranking list, then a statistical approximation is used (as described in the next paragraph).
\end{enumerate}

The results of these steps are: Music entity metadata that connects locally available entities with metadata from Last.fm and Echonest, and a number of similarity connections between locally available artists, retrieved from Last.fm. All the retrieved data resides in the device's memory or storage. In empirical tests, it has been found that as much as 90 percent or more of potential similarity connections cannot be established based on the web APIs' results. That means that up to 90 percent of artists in a music collection have no incoming or outgoing similarity link to another artist in the music collection. This is due to the fact that Last.fm will not output more than 100 entries of similar artists at this time, making it unlikely that artists are contained in each other's similarity ranking lists. It is not possible at this time to query Last.fm specifically for the similarity between an artist X and an artist Y, so the similarity between X and Y is only available when artist Y is in artist X's similar artists list, or vice versa. 
Omitting the definition of any default similarity value in such cases (leaving the similarity between examplary artists X and Y at zero or undefined) could generally be an option. However, the selected 2D-projection algorithm (multi-dimensional scaling algorithm using a spring model) produces undesirable results with this approach (placing artists without a similarity link at completely random positions, often far away from the other artists, confusing the user).
So, a meaningful default similarity for unknown similarities must be defined. A reasonable approach is to use a statistical approximation. An artist Y (who is not in the similarity ranking list of an artist X) must have a lower similarity to X than the least similar artist in the similarity ranking list of artist X. Since we lack any additional data, the expected similarity of artist X and artist Y then equals:\\
\emph{0.5 * [least similarity value in the similarity ranking list of artists of artist X]}.\\
This approximation is applied to all similarity links which point to locally available artists which are not in an other locally available artist's similarity ranking list, instead of a zero or undefined similarity.

\section{Multidimensional Scaling Algorithm}
\label{sec:mds-algorithm}

\subsection{The Given Problem}

The subjective similarity of music, or artists in particular, can be expressed as attributes which are assigned by human beings (their opinions). Such attributes are:

\begin{itemize}
	\item Rhythm
	\item Beats per minute
	\item Mood
	\item Popularity
	\item Genre
	\item Tonality
	\item ...
\end{itemize}

Whatever the assigned attributes are - by comparing attributes of music titles to those of other titles, a similarity measure can somehow be obtained. Every attribute then can be seen as another dimension, and every music title or artist is an object in a space of N dimensions, where N is the number of attributes assigned. Therefore, the problem of laying out music collections in two-dimensional layouts with respect to their subjective similarity can be reduced to a multidimensional scaling problem.
Since the webservice Last.fm provides readymade similarity measures between artists, we can take a shortcut here, and remain oblivious of audio attributes such as rhythm or mood. It has not been publicized how Last.fm generates similarity information based on their users' behaviour, but it can be assumed that they use Collaborative Filtering algorithms ~\cite{Takacs:2007}, or market basket analysis ~\cite{Aggarwal99anew}. Due to the possibility of externalizing similarity data aggregation to a Web Service, a major part of spring model MDS work - the filtering and mapping of different attributes w.r.t. their impact on object similarity - is a solved problem. The similarities that are retrieved from Last.fm can directly be used to calculate the length of the connecting springs.

\subsection{Concrete MDS Algorithm with Optimizations for Execution on Mobile Devices}

As mentioned before, the generation of a 2D-projection of music collections is performed by executing a MDS algorithm using a spring model. In section ~\ref{subsec:2dprojection}, such a basic MDS algorithm was described, and in the same section, optimizations introduced by \cite{Chalmers:1996:LIT:244979.245035} and \cite{Morrison:2003:FMS} have been listed.

In the following, the optimized algorithm will be explained step by step (SeedNodeIterations, NeighborCandidateSetSize, CleanupIterations being predefined constants):

\begin{enumerate}
	\item Select a random set \textbf{seed nodes set} of $\sqrt{N}$ nodes (N = number of all artists available on the device)
	\item Perform a sampled MDS calculation on the initial layout, by performing \emph{SeedNodeIterations} iterations of the following:
		\begin{enumerate}
			\item For each \textbf{current seed node} in the \textbf{seed nodes set} do:
				\subitem Define a \textbf{neighbor set} of nodes which are neighbors of the \textbf{current seed node} - it is initially empty, and will be filled later on. Define a \textbf{displacement forces set} of forces which will be applied at the end of dealing with the \textbf{current seed node} (step 2.b).
				\subitem Generate a \textbf{neighbor candidate set} which contains \emph{NeighborCandidateSetSize} nodes (taken from ALL nodes, not only from the \textbf{seed nodes set}); concurrently, switch objects into the \textbf{current seed node}'s \textbf{neighbor set} if either the \textbf{neighbor set} has not reached its capacity, or if the \textbf{current seed node} has a lower input space distance (looked up in the similarity matrix) than the neighbor node with the highest input space distance (also looked up in the similarity matrix).
				\subitem For every object in either the random subset or the \textbf{neighbor set}, calculate its optimal (should-be) output space distance to the \textbf{current seed node}. The supposed output space distance is determined by the Euclidean distance in the output space (2D), by taking the similarity value retrieved from Last.fm into account.
				\subitem If the optimal output space distance does not match the actual current output space distance, then calculate a force for the \textbf{current seed node} into a certain direction (vector), either attractive or repulsive, and save it into the \textbf{displacement forces set} for application in step 2.b. (Note: Only the current seed node is moved.)
			\item Apply all previously saved forces in the \textbf{displacement forces set} onto the corresponding seed nodes, by displacing each seed node with the sums of all saved force vectors in \textbf{displacement forces set} which belong to the seed node. The result is that every seed node in \textbf{seed nodes set} has been moved by forces enacted on it by other nodes produced in step 2.a. The forces are not applied earlier so that the nodes don't move in the middle of calculating forces of other nodes during step 2.a.
		\end{enumerate}
	\item For every \textbf{node} NOT in the \textbf{seed nodes set} do (with \textbf{processed nodes set} initially containing all objects from \textbf{seed nodes set}):
		\begin{enumerate}
			\item Find the most similar \textbf{processed node candidate} (having the lowest input space distance to \textbf{node} of nodes in \textbf{processed nodes set}) , and choose the least stressed quadrant around it, to place the \textbf{node} there. There are four quadrants surrounding \textbf{processed node candidate}, top-left, top-right, bottom-left, bottom-right. The least stressed quadrant is determined as follows:
				\subitem Vectors are constructed from the \textbf{processed node candidate} to each quadrant's center, with a length corresponding to the similarity between \textbf{node} and \textbf{processed node candidate}, taken from the similarity matrix.
				\subitem Positioning tests are performed by moving the \textbf{node} to the end of each constructed vector, and calculating the stress between the \textbf{processed node candidate} and the temporarily moved \textbf{node}. The \textbf{vector of the quadrant} which yields the lowest stress wins the test (being the least stressed quadrant), and \textbf{node} is assigned the resulting position, which is again the result of adding the \textbf{vector of the quadrant} to the \textbf{processed node candidate}'s position.
			\item Add \textbf{node} to \textbf{processed nodes set}.
		\end{enumerate}
	\item Perform a sampled MDS calculation like in step 2. with \emph{CleanupIterations} iterations, but with all available nodes instead of with nodes in \textbf{seed nodes set}. The purpose is to move grossly mispositioned objects to a better position - very good results can already be achieved with a low value of \emph{CleanupIterations}.
		
\end{enumerate}

\section{Downsides of MDS in this Context}

The advantages of MDS have been described sufficiently in this section. However, there are some downsides to this approach too, some of which are inherent to the problem at hand and which could not be alleviated by using another method.

If the internal structure of the transformed music entity set \textbf{is not expressible in a two-dimensional space without tradeoffs}, the spatial distances between entities will not always depict their similarity correctly. The larger the transformed set is, the more will the entities' 2D distances be different from their expected ("'should be"') distances. It can be viewed as a given that in most cases distances between some entities will not at all match their expected lengths.

The layout produced by the aforementioned MDS method \textbf{will never be the same for two different calculations}, since parts of the algorithm involve randomness 
\footnote{The introduction of randomness in the layout creation algorithm has been necessary, as mentioned before, because it reduces its time complexity from $O(n^3)$ or even $O(n^4)$ to $\mathcal O(N*\sqrt{N})$. Without these optimizations, the algorithm would hardly be able to produce a layout in a mobile app in reasonable time.}.
While the addition of some entities is possible after the layout has been completed, a complete rerun of the algorithm will become necessary if similarities change. This is suboptimal for users because they have to reorient themselves after every major layout change.

The MDS method presented earlier in this chapter relies on definite high-dimensional vectors (from which similarity measures can be derived), which is not available here - in the context of computing similarity based on web sources, it is unlikely that a similarity measure can be found for every entity-entity relationship. Therefore, \textbf{estimations have to be used for unknown similarity relationships}, potentially polluting the calculation with incorrect values.

\section{Removal of Node Overlapping}
\label{sec:removal-node-overlapping}

As the reader may have noticed, the positions of objects computed as described in Section ~\ref{ch:computation} don't respect any aesthetic criteria. ~\cite{Li:2005} gives an overview of such criteria, three of which also apply to graphs produced by MDS:

\begin{itemize}
	\item Minimization of the area taken up
	\item Minimization of total object distance
	\item Aspect ratio close to or matching the specification
\end{itemize}

However, an even more important aesthetic factor in the perception of graph drawings is the \textbf{removal of node overlappings}. The computational layout method applied in this thesis - spring model MDS (see Section ~\ref{ch:computation}) - can not give any heed to overlapping nodes (or margin between them), since nodes are positioned as points, not 2D objects. However, the objects do need to be displayed as 2D shapes (e.g. as rectangles, circles, images,...), and therefore the resulting layout can (and in many cases, will) contain overlappings.
As this may impact the perception of the graph by humans in a negative way (objects may even be completely hidden by other objects), a way has to be found to eliminate overlappings as reliably as possible, without affecting the conveyed information of object similarities too much.

The authors of ~\cite{journals/vlc/MisueELS95} therefore define the concept of the Mental Map, meaning that a graph has properties which should be maintained if transformed: Orthogonal Ordering, Proximity Relations, and Topology. An algorithm called Force Scan (FS) is then proposed to remove node overlappings while preserving the Mental Map of a graph. FS operates by first moving nodes horizontally, maintaining their horizontal ordering, and then moving nodes vertically, maintaining their vertical ordering. Several improvements to FS have been discussed (~\cite{Hayashi:1998:LAP:647550.728930} ~\cite{Huang03force-transfer:a} ~\cite{Li:2005}), while the algorithm in ~\cite{Huang03force-transfer:a} - \textbf{Force Transfer (FT)} - seems to be the most suitable for this thesis' use case, as it is of lower computational complexity than FS and produces pleasing layouts.

The FT algorithm identifies clusters of overlapping nodes, and then performs transfer scans on them: left-to-right, right-to-left, upside-to-downside, and downside-to-upside. During the scans, the nodes in each cluster are moved such that the clusters become free of overlaps. Multiple iterations may be necessary, as the separation of one cluster may cause nodes to move on top of nodes from other clusters - thereby forming new clusters. Finally, the layout eventually is overlap-free, preserving the aforementioned mental map of the graph.

The aforementioned methods of overlap-removal are complex (considering the need to form clusters), so the author decided to employ a more basic approach, but very similar to the previous algorithms. The graph canvas is virtually emptied, and re-filled one by one with the originally contained artist nodes, while making sure that a newly added node does not overlap with any others. The algorithm is described in the following listing:

\begin{enumerate}
\item Remove all nodes from the graph structure.	
\item For every node ("'subjectNode"') do:
	\begin{enumerate}
		\item For every node ("'alreadyPositionedNode"') in the set of already positioned nodes do:
		\item If the distance between subjectNode and alreadyPositionedNode is beloww a
		threshold, create a vector from alreadyPositionedNode's center to subjectNode's center, set
		its length such that they don't overlap, and displace subjectNode by it.
		\item If subjectNode still overlaps with another node, step to 2.a.
		\item Add subjectNode to the set of already positioned nodes.
	\end{enumerate}
\end{enumerate}

The result of this algorithm is a graph state which is free of node overlappings, with a predefined node rectangle size.

\section{Summary}

The matching of music entities from Last.fm and Echonest has been described as a simple and straightforward process with the result that locally available music entities (artists, albums, tracks) are augmented with metadata from these two web APIs. Artist similarity measures are retrieved from Last.fm by querying the web API for artist similarity lists, and it was also mentioned how calculation of expected values can provide meaningful approximations for unknown similarities between artists.

The concrete multi-dimensional scaling algorithm (using a spring model) was then presented in-depth, incorporating optimizations (sampling, interpolation) which make execution of 2D layout generation for a big number of objects possible on a current-generation mobile device. Finally, downsides of the application of MDS for the generation of layouts were found - layouts would hardly ever look the same after two separate calculations, and estimations of unknown artist similarities may pollute the resulting layout.