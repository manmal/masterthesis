This master's thesis was written with the intention of finding a robust way of displaying the similarity of artists in a collection of music titles, in the context of mobile devices. It presents a viable path of creating algorithms, and subsequently, a mobile app for Android which is able to demonstrate artist similarity visualization effectiveness, and a user study with ten persons to evaluate the resulting prototype (called "'Andromeda"').
Today, existing music collection visualizations are manifold and can be built dedicatedly for certain use cases of music playback or browsing software, due to the low development effort needed. Recent developments in music distribution - the paradigm changes from music on physical media, to file-based distribution over the internet, to ad-hoc streaming from a single company's servers - have changed music software users' environment and usage patterns greatly. Previously, only hundreds or thousands of music titles (tens or hundreds of artists) were available to the typical user, but now subscribers of music services have access to millions of titles (many thousands of artists). While in the previous environment it was often possible to form an almost complete mental model of a collection (e.g., users are able to recall all the artists on their hard-drive), this is now impossible for the average user, due to memory and time constraints. It is therefore well possible that existing collection visualization models are not at all suited for presenting such massive amounts of data. Music services like Spotify include workarounds like similar artist linking, but these functions have been added to an ordinary list-based browsing system, and can thus not be fully integrated into the visualization. The challenge is to find a visualization which is both intuitive for the user and comes to terms with presenting only a reasonable excerpt of the music collection, while maintaining the possibility to navigate to other excerpts without losing the current context. This thesis takes the first step towards such a solution by concentrating on finding, implementing, and testing a visualization which shows artists based on their respective similarities to each other. Further research will have to find a solution which is able to extend this concept so that it can display thousands or millions of artists (and potentially also albums and songs), while taking into account their semantic relationships (like similarity data computed by services like Last.fm, as has been used in this thesis).

Chapter ~\ref{ch:relatedwork} presents related work, providing insights into topics motivating and substantiating the topic of this thesis. Feature extraction from music bitstreams is mentioned here, which is a possibility of establishing similarity between music titles, and subsequently, artists. However, it is disregarded for the scope of this thesis because it involves computational effort that is not yet manageable on mobile devices. Instead, data which is readily available from the social music web service Last.fm is used to retrieve similarity ranking lists for artists.Through the creation of distance matrices from the gathered similarity metrics, multi-dimensional scaling by application of a spring model can be applied to solve the problem of laying out artists in a way that their positions approximate their similarities to each other.

Subsequently, in Chapter ~\ref{ch:computation} the scenario and research objectives of this thesis are defined. Research of proper methods to create a 2D visualization that use artist similarity as implicit ordering metric, the implementation as a prototype, and the execution of a user study are determined as objectives. Hypotheses are to be tested by the user study, and the success of the visualization is depending on this outcome. Furthermore, the presentation mode of the visualization is defined to be two-dimensional, and the 2D layout calculation is set to multi-dimensional scaling and a post-processing step in the form of a node overlap removal algorithm. At this point also the exact sequence of processing and calculation steps in the Andromeda prototype App is defined.

In Chapter ~\ref{ch:computation}, the retrieval of artist similarity based on web sources like Last.fm is described, and Last.fm is used as the only source for similarity data further on. A way of dealing with uncertainty (web source does not contain a certain artist-artist similarity) is found. Here, a simple probabilistic assumption is used - an artist is assumed to have a similarity strength to a certain artist which is half as strong as the least known similar artist. After the applicability of multi-dimensional scaling (by application of a spring model) to the problem domain of 2D-projection has been confirmed, the concrete multi-dimensional scaling algorithm is described in length. Similar to a physical system of metal springs pulling at the system's objects from different directions, the objects in this MDS algorithm are pulled or pushed towards other objects. After the discussion of MDS algorithm specifics, optimizations for the execution in mobile devices are presented, which are necessary due to the limitations of mobile processing power. Finally, the downsides of applying MDS in the context of this thesis are considered - in a large music collection, the distances between artists in the layout would not necessarily resemble their actual similarity. This is due to the problem that real multi-dimensional data which is reduced to two dimensions will introduce compromises and tradeoffs in the positioning of objects. So, clusters of objects which exist in higher-dimensional data (such as artists in a certain genre) might not be reflected in the resulting 2D layout. This problem should be tackled by further research on how to preserve clusters in the resulting 2D layout.
Another algorithm presented here has the task of removing overlappings between the artists (nodes) in a layout. After the discussion of several methods which have been researched, a more basic approach for removal of overlappings is presented, which tries to preserve the layout's implicit structure. Further research will have to tackle the shortcomings of this algorithm, as is described in section ~\ref{sec:removal-node-overlapping}.

After the discussion of artist similarity computation, details of its visualization are described in Chapter ~\ref{ch:visualization}. The user interaction model is a basic one, similar to other established apps or software in other domains like web sites. Breadcrumbs are introduced as a means of hierarchical navigation between screens.

In Chapter ~\ref{ch:implementation}, the implementation of a prototype app ("'Andromeda"') for the Android OS is described in length. A verbose list of the algorithm sequence in Andromeda is given, giving an overview of the flow and processing of information as it is retrieved from web sources and turned into a visual layout. Subsequently, steps of the multi-dimensional scaling and node overlap removal algorithms are illustrated by actual Andromeda screenshots. Some intermediate steps show how MDS changes the positions of an initial subset of artists, then how artists are added to the layout, and finally how the overlap removal algorithm pushes artists apart in a semantics-preserving way. Information gathering from web sources is then discussed, including music metadata and artist-to-artist similarity data. Pseudocode algorithm listings illustrate parts of the concrete implementation of MDS and the removal of overlappings. A short introduction to Andromeda's OpenGL and animations gives insights on the implementation parts which affect the user experience and interaction model the most.

In order to evaluate Andromeda's usability and performance (and thus, evaluate the implemented visualization), a user study is planned, executed, and analyzed. A set of five hypotheses is postulated, among them the claim that the 2D visualization based on artist-similarity results in the better forming of a mental model, and thus in better memorization of artists in the collection. Hints to the acceptance or rejection are gathered by comparing Andromeda (in two versions) with two other mobile music applications. As test metrics, a collection of tasks are defined which each user is asked to perform, and each task must be performed with every visualization. Additionally, several questionnaires, among them standardized AttrakDiff 2 and System Usability Scale tests, are employed. With the help of these, test participant data and usability metrics are gathered. The study is performed with a sample of 10 persons, of varying ages and professions. After the presentation of important data excerpts, statistical key figures, and questionnaire feedback, the study results are analyzed. Since the low participant count does not allow for meaningful stochastic investigation, the data is analyzed for hints towards acceptance or rejection of the hypotheses instead. The only hypothesis which seems to be supported by the experiment data is the better memorization of artists through the visual arrangement of artists by similarity. However, all the other hypotheses seem to be rejected: Faster finding of certain artists by name (H2), faster re-finding of artists (H3), faster finding of artists related to a predefined subject artist (H4), and faster finding of 3 artists by mood (H5) were all hinted against by the gathered data.

 It is concluded that Andromeda enables better mental model forming of music collections than other apps.

\section{Outlook}

As mentioned before, the findings of this thesis seem to imply that the proposed visualization at least allows easier memorization than common mobile music apps. It is also inferred that this is due to the easier or better forming of a mental model of the presented music collection. Subjective artist similarity visualization is a promising concept, especially in the context of ever-growing music collections which cannot any longer be presented as a whole, but only in the form of excerpts. Multi-dimensional scaling, as presented in this thesis, is a sound algorithm for the problem at hand, but the produced layout's significance lessens with a growing number of objects. Research should concentrate on the alleviation of this problem - it might be worth considering a 3D layout, if a suitable touch-based navigation model can be found.

Further research should:
\begin{itemize}
  \item Find a way to display related artists during artist discovery mode (described in \ref{sec:artist-similarity-vis}) while concurrently visualizing their similarity to the seed artist (currently, similarity is not taken into account). This can be achieved either by changing their proximity of the related artists to the seed artist, or by adding other visual clues (color coding, opacity changes,...). If proximity shall indicate similarity, then the force-directed layout algorithm which is currently used will have to be modified accordingly.

  \item Investigate better performing options (which better preserve spring model system stress) of algorithms for the removal of node overlappings in 2D layouts, as discussed in section ~\ref{sec:removal-node-overlapping} and ~\ref{subsec:libraryvis}. The currently used algorithm for the removal of node in Andromeda is cosmetically pleasing, but does not accurately preserve the layout produced by the MDS (by application of a spring model) 2D-projection algorithm.

  \item Find statistically significant evidence of the hypothesis whether the 2D-projection implemented in Andromeda allows easier forming of mental models, by executing a large-scale user study (H1 in Chapter ~\ref{ch:userstudy}).
\end{itemize}