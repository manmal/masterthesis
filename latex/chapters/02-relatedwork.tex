In this section, the reader will be introduced to the proceedings in science which are
relevant or related to this thesis. They are grouped into the following topics of interest:

\begin{itemize}
	\item \textbf {Motivation for this thesis} - Provides an explanation on why the chosen 
		  topic of this thesis is generally of interest, and why similarity measures in music are feasible
		  (given optimal circumstances).
	\item \textbf {Features of digital music} - Gives an overview of existing methods of 
		  feature extraction which are purely based on computational methods.
	\item \textbf {Subjective music similarity computation} - Lists literature which is related
		  to this thesis' problem of computing the subjective similarity of artists, which is
		  based on subjectiveness as experienced by humans.
	\item \textbf {Visualization of artist similarity} - Provides an overview of existing 
		  methods and models of visualization of music similarity (or, in this case, artist similarity).
%	\item \textbf {Drawing and optimization of graphs based on multidimensional scaling} - Gives an
%		  overview of literature dealing with the technical and mathematical background of 
%		  multi-dimensional visualization on two-dimensional output devices.
\end{itemize}

\section{Motivation for the Topic of This Thesis}

Music is an integral part of the daily life in nearly all societies, and the list of published titles 
is growing every day. As huge amounts of data tend to be hard to digest, ontologies have to be created, 
by which music can be categorized in a hierarchical fashion. Aside from the author's motivation of 
choosing the topic of this thesis, a great interest in music classification can be observed in scientific 
literature. This is related to the problem that the categorization of arbitrary music titles
is neither implicit nor trivial.
Serving the demands of Electronic Music Distribution (EMD), the authors of \cite{pachet:02g} elaborate
on the feasability of music similarity measures. It is found in \cite{pachet:02g} that the introduced
similarity measure (timbre similarity) combined with other measures can yield interesting results. It is
also mentioned that the interpretation of experimental results in the field of music similarity is challenging
due to the subjective demands. 

It is clear that even the best-educated music experts could hardly agree on 
any distinct similarity measure between two music titles, due to the implicit fuzziness of subjective measures.
It can be assumed that it is rare that two humans would agree on the same similarity between music files
if they vote independent of each other.

\section{Features of Digital Music}

As opposed to subjective artist similarity, there are music features or measures which can be retrieved by
purely computational approaches. In the field of audio feature extraction, a wide range of classifiers
(feature extractors) has been created. These classifiers in many cases run a bitstream analysis of a digitally
stored music file and extract one or more reproducible measures characterizing the file. 
Interestingly, it is confirmed in \cite{LID_05ismir} that the use of psycho-acoustic enhancements before
feature extraction improves the classification accuracy significantly. It can be concluded that the outcomes of
audio feature extraction are influenced by many factors which are not always intuitive.
As has been mentioned previously, most audio classifiers analyze the bitstream of music files - however,
the bitstream is only one dimension of a piece of music, if we regard it as a multidimensional object. For example,
it is also possible to analyse the lyrics, as has been done in \cite{DBLP:conf/ismir/MayerNR08}.

\section{Subjective Music Similarity Computation}

Subjective similarity, as the author understands it, expresses human opinions on a certain object. As previously
mentioned, it is obvious that humans will hardly agree on attributes of music, and the same person might even make
different statements in the course of time, depending e.g. on her mood. The following applies to both artist 
similarities and music file similarities, since the former may be constructed from aggregations of the latter 
(it has to be noted at this point that many artists tend to produce music from multiple genres, thus making
an artist-to-artist-comparison difficult or even infeasible).
In article \cite{Ellis02thequest} it is found that it is doubtable that a common ground truth for subjective
artist similarity even exists, because of the inhomegeneity of measures made by the involved users. It can be
deduced that a meaningful model of subjective music similarity will in most cases only resemble a compromise
between different stakeholders.
As inferred from \cite{Berenzweig03alarge-scale} and \cite{mcfee09_hesas} there are different approaches to 
retrieving a model of subjective similarity for a given set of music files, which include:

\begin{itemize}
	\item Conduction of surveys with end users
	\item Opinions of experts
	\item Co-occurrence of files in end users' libraries or playlists
	\item Data mining of text in web sources, as performed in \cite{Whitman02inferringdescriptions}
	\item Leveraging data gathered by social music services
\end{itemize}

As it is intended by this thesis to provide a concept for a fast and fully automatic approach to similarity
measuring, we will concentrate on the last approach, the usage of data provided by social music services.
Hybrid computation methods, such as the method described by \cite{mcfee09_hesas} (combining acoustic 
features with text excerpts and tags retrieved from online services) turn out to be hardly feasible on a 
mobile device because of performance requirements. It is assumed by the author that for a rough estimation 
of music file or artist similarity, the data provided by social music services (as opposed to hybrid 
approaches) is sufficiently meaningful, as their daily user base is in the millions and still growing.

Apart from the source of similarity metrics, also the scope of computation has to be given thought to.
Depending on a user's scenario, the user might want to explore her own music collection, or she might want
to discover music similar to her own. In the following subsections, both cases are considered.

\subsection{Similarity Computation for Collections of Music}

In order to provide a meaningful, semantic overview of a collection of objects, data must be available 
or generated for all objects in the collection, or for most of them. 

As has been mentioned before, \textbf{extraction of features} of music files may be used to gather metrics
about the analysed files. The data retrieved may then be combined into a feature vector of N dimensions,
where N is the number of features. When all files in the analysed music collection have been classified in
this way, a \textbf{self-organizing map} can be trained with the resulting feature vectors, mapping 
vectors with small Euclidean distances spatially near to each other \cite{RAU_02ismir}. This results
in a map where elevation represents the frequency of vectors in certain areas - thus, a high elevation
at a point on the map means that the feature vector attached to the point is similar to a big number of music
files. It is obvious that self-organizing maps are well suited for visually clustering music files by their 
features. However, it must be noted that this approach is only feasible if a huge amount of music metadata 
can be retrieved, either by feature extraction or other methods like text mining. Therefore, self-organizing 
maps can be considered not being quite suitable for the goals and scope of this thesis.

An alternative to feature extraction in this context is the construction of a 
\textbf{similarity matrix} containing all objects of the analysed music collection. A similarity matrix contains
the similarities (or rather the dissimilarities) of all items to each other, in the form of distances.
Intuitively, the measured item-item distance is higher if they are more dissimilar to each other (a distance of
zero expressing equalness between items). Data sources for similarity matrices describing music include 
(as described earlier in this chapter):
\begin{itemize}
	\item Surveys, playlist co-occurrences, user collection co-occurrences, web text mining,... 
	\item Similarity rankings or measures from public Web APIs
\end{itemize}
As mentioned before, we will concentrate on the latter for complexity reasons. The task of building up a 
similarity matrix for a collection of music titles would then be reduced to a number of Web API calls, mapping the
resulting distances or rankings to the objects in the collection. Similarity matrices can then be processed
to find a suitable two-dimensional representation in Euclidean space, where the spatial distance between
objects represents their similarity, or their dissimilarity respectively. This process is called \textbf{multi-dimensional scaling (MDS)}, and it has found broad adoption in scientific and industrial applications. 

One of the most common multi-dimensional scaling techniques is the adoption of a spring model, which emulates the physical behaviour of steel rings connected by metal springs \cite{Morrison:2003:FMS}. To continue with the analogy, MDS starts off with the steel rings at random positions, and in multiple iterations tries to satisfy the springs' forces. A spring's force is usually proportional to the discrepancy of two objects' high-dimensional distance and their low-dimensional distance. The fitness of the model (all steel rings being at their optimal position, considering all connected springs) is described by a stress function. The lower the stress function's output, the better the new low-dimensional model suits the original high-dimensional model. In most cases a perfect match between high-dimensional and low-dimensional Euclidean distances (stress function output = 0) is not possible, and thus for the MDS calculation to terminate, sensible termination criteria have to be defined - e.g., if the velocity of objects moving at each iteration falls below a predefined threshold, the algorithm terminates and the resulting low-dimensional representation is accepted as being "'good enough"'.

Unfortunately, common spring model (or: metric distance) MDS is inflexible in the sense that the whole computation has to be performed all over again if slight changes to the data set occur \cite{Morrison:2003:FMS}. 
Therefore, a computationally advantageous and more flexible approach to multi-dimensional scaling has been presented in \cite{Morrison:2003:FMS}, combining \textbf{MDS with sampling and interpolation}. As opposed to common multi-dimensional
scaling, this hybrid methodology starts off with a random sample of size $\sqrt{N}$, where N is the number of objects contained in the dataset. After the spring model computation described above has been performed on the subset, the rest of the data set ($N - \sqrt{N}$ objects) is integrated into the low-dimensional model by an interpolation process which is described in detail in \cite{Morrison:2003:FMS}. It has been found in the article that this combination of algorithms improves greatly on the accuracy of the model (i.e., a lower stress function output) and offers a sub-quadratic run time of $\mathcal O(N*\sqrt{N})$.

\subsection{Discovering Objects Similar to a Given Object}

After the elaboration of means of exploring the semi-static collection of objects in a user's library, heed must be given to the recommendation of unknown objects. Equipped with similarity data retrieved from various web APIs it is not only possible to compute relations of objects within a library, but also to find related objects which are currently not present in the library. It is clear that the same algorithms which have been previously described would compute usable outcomes by simply adding unknown objects to a library's representation (e.g. spring model MDS); yet, it can be assumed that in most cases only one object in a library is the starting point of a search for similar objects. This renders a big part of the library's representation irrelevant for this use case - consider a user searching for interprets similar to "'The Beatles"' - most likely, she will not be interested in how these unknown interprets relate to other bands in her library. Also, integrating such previously unknown objects into the representation of a big library will be computationally and query-wise infeasible, as dissimilarity distances to all objects in the library have to be determined. It must be noted that the representation for unknown object recommendation can only be a crude approximation (because of the previously described volatility of subjective similarity), and for this reason not only numerical measures, but also similarity rankings are considered sufficient for this use case.

A derivation from a previously proposed method can be considered here: Suiting the requirements well is a spring model MDS approach applied to a small dataset consisting of the starting point object (in the example being "'The Beatles"') and previously unkown objects which are most similar to it (retrieved via web APIs). The size of such a dataset would presumably peak at 30-40 objects, making common spring model multi-dimensional scaling computationally feasible. Naturally, the sampling/interpolation approach described in the previous subsection would also apply here, further decreasing the computation time.

However, a computationally less expensive algorithm for such means has been proposed in \cite{Marshall:2010}, consisting of a fusion of similarity rankings from various social music services. In this article it is demonstrated that various methods of embedding (fusing) similarity rankings from online services can provide different meaningful similarity models, some of which give more weight to unknown artists. Intuitively, this methodology is able to compute a ranked list of similar objects, based on multiple sources for greater reliability, in a customizable way. The fusion methods reach from rank-average to concordet-fusion (unweighted directed graph). Three major benefits speak in this approach's favour over global numerical similarity measurements:

\begin{itemize}
	\item \textbf{Potentially insignificant computation times} while preserving a stable similarity ranking very well suited for mobile end users.
	\item \textbf{Simple but effective customization} achieved by easily exchangable fusion algorithm components.
	\item \textbf{Reducing the number of web API queries to a minimum} greatly reduces the overall number of network requests, making the method even better suited for mobile usage.
\end{itemize}

It must be noted at this point that the rank fusion algorithm is neither able to define distances between arbitrary objects in the dataset - only a dissimilarity ranking between the starting point object (e.g. "'The Beatles"') and the remaining objects is obtained - nor is it able to force the inclusion of objects (from a local library) in the ranking. This algorithm depends fully on the objects provided by external sources, meaning that if the starting point object is not contained in external sources, no meaningful result can be obtained. However, the author considers the amount of objects which can be obtained from third party web APIs sufficient for the algorithm to perform well for most of all objects in a typical user's library.

\section{Visualization of Artist Similarity}

The mode or fashion of data visualization can be considered a crucial aspect of interfaces for humans (i.e., graphical user interfaces). Today, humans' ability to apprehend information presented to them is limited in several ways, some of which are physical, and some of which are of psychological nature. Some of these limiting factors include:

\begin{itemize}
	\item Restriction of short-term memory,
	\item Limited power of concentration,
	\item Narrow attention span (especially while using mobile device),
	\item Color blindness,...
\end{itemize}

Therefore, it is desirable to give heed to the choice of visualization method to achieve optimal apprehension results, without hindering information understanding through visualization errors.
Since the field of music and music collection visualization is broad and not all algorithms can be presented within this thesis, the author decided to select only the field of two-dimensional collection visualizations for further investigation. It must be noted that several fields of visualizations are left out of the scope here, including:

\begin{itemize}
	\item \textbf{Abstract visualization as an artform} - Certain artforms try to make music more tangible by creating matching images, as in the movie "'2001 - A Space Odyssey"' by Stanley Kubrick, or in works by the demo scene in Germany \cite{Scheib:2002}.
	\item \textbf{Realtime computed images as abstract visualization} have become common components of many desktop audio players, presenting the user with animated images (fractals, 3D-animations,...) which somehow resemble certain features of the currently played music.
	\item \textbf{3D environments resembling music content} give users the ability to roam through a virtual space similar to the way they interact with the physical world, as described in \cite{Dittenbach:2007}.
\end{itemize}

The scope of this thesis confines itself to the visualization of music tracks as objects, relating these objects to each other, and disregarding their real-time aspects (i.e., not generating any visualizations during playback).
Intuitively, the computation and visualization of those relations (also, the quality of relations, e.g. ranking or dissimilarity distances) are closely related to each other. In some cases, a certain mode of computation of object relations more or less forces or forbids the usage of certain visualization approaches. Therefore, the presented modes of visualization are at least closely related to their computational counterparts from the previous subsection.

\subsection{Visualization of Collections of Music}

Previous work has shown that \textbf{self-organizing maps (SOM)}, which are in this context also called "'islands of music"' are well suited as visualizations of related music objects \cite{Cooper:2006:VAM}. This methodology depends on raw audio stream analysis (performed by aforementioned feature extraction algorithms), and subsequently displaying them on an elevation-map, similar to a geographical map. Proof-of-concepts have been successfully implemented, as has been demonstrated in \cite{NeuDitRau_05ismir}, featuring the PlaySOM. The information such self-organizing map visualizations want to give the user is: There are clusters of similar pieces of music in the provided collection, and within one cluster the contained music files most similar to each other. Additionally, each cluster has its own weight vector which can be used to add semantic height annotation to the map - for example, clusters whose objects contain a high tempo can be marked "'high"' (as opposed to clusters with slower music being marked as "'low"'), generating a corresponding height profile.

Another broad group of (dis-)similarity visualizations is made up of \textbf{force-directed graph layouts}. They all consist of nodes (music objects) and edges (relations between objects). Additionally, the distances (edge lengths) between nodes approximate a function over the previously determined dissimilarities between the music objects.
As has been described in \cite{gansner:1998}, the application of pseudo-physical forces on an undirected graph provides for a improvement of the graph layout. This is achieved by adding attractive or repulsive forces to all nodes in the graph, such that nodes push away from or attract each other. As long as there is energy left in a graph (i.e., there are objects which are not in their optimal position), the nodes are moved in a way that satisfies the applied forces. The authors of \cite{Muelder:2010fk} have described and experimented with several graph-based layouts, and among them was a force-directed layout algorithm called LinLog \cite{noack:2003}, which has been found to deliver the most aestethic results. However, a computational model which might be more suitable for the calculation on mobile devices is presented (among other methods) in \cite{Kobourov04}.

The forces in a force-directed graph layout can behave like springs connecting nodes, and for this reason a subset of force-directed graph layouts is called \textbf{spring model}, which has been described in length in the previous subchapter in the context of multi-dimensional scaling (MDS). The calculation of a spring model's layout is very tightly coupled with the overall MDS computation, even in hybrid approaches \cite{Morrison:2003:FMS} - in the case of MDS in combination with spring models, the visualization approach cannot be cleanly separated from the computation approach.

Other graph layouts which pose options for music collection browsing include \cite{Muelder:2010fk}:

\begin{itemize}
	\item Principal Component Analysis (PCA) layouts 
	\item Tree map layouts 
	\item Space filling curve layouts
\end{itemize}

\subsection{Visualizations for the Recommendation of Unknown Objects}

As has been discussed in the previous subchapter, the computational approaches for the recommendation of new objects can be either very similar to the computation of ordinary collection visualization, or they can use more simplified rank-based models. The former are clearly covered properly by the broad range of previously described visualization methods for collections of objects.
On the contrary, visualization possibilities for rank-based computation models are not as manifold, due to the fuzzy dissimilarities between objects - a ranking can not be used to acquire deterministic object distances. However, since the goal of the visualization of such a ranking is to provide a very rough overview to the user, a deterministic visualization is not necessary. It can even be considered to omit the ranks, and just display these objects as relations of the same relevance, as has been done by the authors of \cite{DBLP:conf/webist/SarmentoGCO09}. It seems that also for this use case, a force-directed graph layout provides for the most aestethic results \cite{DBLP:journals/spe/FruchtermanR91}. Additionally, such layouts can execute their self-optimization in realtime while being presented to their user without affecting the user experience in a negative way, as is shown by \cite{url:tuneglue}. Additionally, the nodes in such layouts are user-manipulable in realtime.

% \section{Drawing and Optimization of Force-Directed Graph Layouts}


\section{Summary of this Section}